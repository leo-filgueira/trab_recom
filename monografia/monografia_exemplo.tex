\documentclass[12pt,a4paper,header]{abnt}

\usepackage[brazil]{babel}        
\usepackage[utf8]{inputenc} 

\usepackage{hyperref}
\usepackage{breakurl}
\usepackage{bookmark}
\usepackage{amsmath,amssymb,amsfonts,undertilde}
\usepackage{graphicx}
\usepackage{subfigure}
\usepackage{fancyhdr}
\usepackage{siunitx}


%%%%%%%%%%%%%%%%%%%%%%%%%%%
%Teoremas, definicoes, etc%
%%%%%%%%%%%%%%%%%%%%%%%%%%%

\newtheorem{thm}{Teorema}[section]
\newtheorem{cor}[thm]{Corolário}
\newtheorem{lem}[thm]{Lema}
\newtheorem{defi}[thm]{Definição}
\newtheorem{exe}[thm]{Exemplo}
\newtheorem{prop}[thm]{Proposição}

\renewcommand{\ABNTchapterfont}{\bfseries}
\renewcommand{\ABNTsectionfont}{\bfseries}

\fancypagestyle{logouff}{%
	\renewcommand{\headrulewidth}{0pt}
	\fancyhead{}
	\fancyhead[R]{\includegraphics[width=0.7\textwidth]{logoUFF.pdf}}% Your logo/image
  	\setlength{\headheight}{30pt} 
  	\setlength{\headsep}{2cm}
}





\begin{document}

%%%%%%%%%%%%%%%%%%%%%%%%%%%%%%
%colocar aqui o nome do aluno%
%%%%%%%%%%%%%%%%%%%%%%%%%%%%%%
\autor{Leonardo Filgueira}


%%%%%%%%%%%%%%%%%%%%%%%%%%%%%%%%%%%%%
%colocar aqui o título da monografia%
%%%%%%%%%%%%%%%%%%%%%%%%%%%%%%%%%%%%%
\titulo{Sistemas de recomendação usando o software R}


%%%%%%%%%%%%%%%%%%%%%%%%%%%%%%%%%%%
%colocar aqui o nome do orientador%
%%%%%%%%%%%%%%%%%%%%%%%%%%%%%%%%%%%
\orientador{Luciane Ferreira Alcoforado}  


%%%%%%%%%%%%%%%%%%%%%%%%%%%%%%%%%%%%%%%%%%%%%%%%%%%
%colocar aqui o nome do co-orientador, caso exista%
%%%%%%%%%%%%%%%%%%%%%%%%%%%%%%%%%%%%%%%%%%%%%%%%%%%
\coorientador{Rodrigo Otávio de Araújo Ribeiro}


%%%%%%%%%%%%%%%%%%%%%%%%%%%%%%%%%%%%%%%%%%%%%%%%%%%
%colocar aqui a data da apresentacao da monografia%
%%%%%%%%%%%%%%%%%%%%%%%%%%%%%%%%%%%%%%%%%%%%%%%%%%%
\data{}



%%%%%%%%%%%%%%%%%%%%%%%%%%%
%nao mexer até a linha 180%
%%%%%%%%%%%%%%%%%%%%%%%%%%%


\comentario{Monografia apresentada para obtenção do grau de Bacharel em Estatística pela Universidade Federal Fluminense.}

\instituicao{Departamento de Estatística \par Instituto de Matemática e Estatística \par Universidade Federal Fluminense}

\local{Niterói - RJ, Brasil}

\capa

\vspace{10cm}


%folha de rosto
%--------------

\begin{titlepage}

\thispagestyle{logouff}

\vspace{2cm}

\hspace{.2\textwidth} % posicionando a minipage
\begin{minipage}{.7\textwidth}

\begin{flushright}

{\large \bf \ABNTautordata} \\[3cm]

{\Large \bf \ABNTtitulodata}\\[3cm]

{\bf Trabalho de Conclusão de Curso}\\[1cm]

\end{flushright}

\begin{espacosimples}

\ABNTcomentariodata

\end{espacosimples}

\vspace{1cm}

\hfill Orientador: Prof. \ABNTorientadordata

\end{minipage}

\vspace{7cm}

\begin{center}

\ABNTlocaldata

\ABNTdatadata

\end{center}

\end{titlepage}


%ficha catalografica
%-------------------
\newpage
\null
\vfill

%\begin{left}
\fbox{
% \includegraphics[scale=1]{ficha_catalografica.pdf}
}
%\end{center}
\vspace{1cm}

%folha de aprovacao com as assinaturas - Editar os membros da banca
%------------------------------------------------------------------
\begin{folhadeaprovacao}

\thispagestyle{logouff}

\hspace{.2\textwidth} % posicionando a minipage
\begin{minipage}{.7\textwidth}

\begin{flushright}

{\large \bf \ABNTautordata}\\[1cm]

{\large \bf \ABNTtitulodata}\\[1cm]

\end{flushright}

Monografia de Projeto Final de Graduação sob o título \textit{``\ABNTtitulodata''},
defendida por \ABNTautordata~e aprovada em \ABNTdatadata, na cidade de Niterói,
no Estado do Rio de Janeiro, pela banca examinadora constituída pelos
professores:

\begin{flushright}

\begin{espacosimples}

%assinatura




%%%%%%%%%%%%%%%%%%%%%%%%%%%%%%%%%%%%%%%%%
%Preencher os dados dos membros da banca%
%%%%%%%%%%%%%%%%%%%%%%%%%%%%%%%%%%%%%%%%%

\vspace{2cm}
\noindent\rule{8cm}{0.4pt}\\
{\bf Profa. Dra. Luciane Ferreira Alcoforado}\\
Departamento de Estatística -- UFF\\


\vspace{2cm}
\noindent\rule{8cm}{0.4pt}\\
{\bf Prof. Dr. Steven Dutt Ross}\\
UNIRIO\\


\vspace{2cm}
\noindent\rule{8cm}{0.4pt}\\
{\bf Prof. Dr. Nome do 2o membro da banca}\\
Instituicao do 2o membro da banca\\

\end{espacosimples}

\end{flushright}

\vspace{2cm}
\hfill Niterói, \ABNTdatadata

\end{minipage}


%\vfill \hfill Niterói, \ABNTdatadata \hspace{1cm} 


\end{folhadeaprovacao}




%%%%%%%%%%%%%%%%%%%%%%%%%%%%%%%%%%%%%%%
%escreva aqui o resumo do seu trabalho%
%%%%%%%%%%%%%%%%%%%%%%%%%%%%%%%%%%%%%%%
% \begin{resumo}
% % Aqui entra o resumo da monografia.
% 
% \vspace{1cm}
% \noindent Palavras-chaves: 
% %Aqui entram as palavras chaves.
% 
% 
% \end{resumo}
% 
% 
% 
% %%%%%%%%%%%%%%%%%%%%%%%%%%%%%%%%
% %escreva aqui a sua dedicatória% (opcional)
% %%%%%%%%%%%%%%%%%%%%%%%%%%%%%%%%
% \chapter*{Dedicatória}
% % Aqui entra a sua dedicatória. 
% 
% 
% 
% %%%%%%%%%%%%%%%%%%%%%%%%%%%%%%%%%%
% %escreva aqui seus agradecimentos% (opcional)
% %%%%%%%%%%%%%%%%%%%%%%%%%%%%%%%%%%
% \chapter*{Agradecimentos}
% % Aqui entram os agradecimentos. 
% 


\tableofcontents{}
\listoffigures
\listoftables



\chapter{Introdução} \label{cap:introducao}

A partir do aumento de informação disponível com a popularização da Internet e com a possibilidade de armazenar essas informações, surge o desafio de lidar com este grande conjunto de dados\cite{isinkaye2015recommendation}. Este aumento de informações desafia o site, como lojas on-line, que recebe todas as informações dos usuários que visitam o endereço, mas também pode se tornar um problema para o usuário que, diante da grande quantidade de produtos disponíveis para compra, pode levar muito tempo para achar o produto desejado\cite{mild2002collaborative}.

Sistemas de recomendação são técnicas de \textit{machine learning} (aprendizado de máquina) que filtram um grande conjunto de dados, tendo como base informações dos usuários e itens\cite{takahashi2015estudo}. A partir dessas técnicas são previstas as notas que os usuários dariam a determinados itens, que podem ser dos mais variados tipos, e, para um indivíduo, recomenda-se o(s) item(ns) que obtiveram uma nota prevista maior\cite{shapira2011recommender}. Os sistemas de recomendação têm como objetivo recomendar itens que interessariam aos usuários\cite{melville2011recommender}, beneficiando o usuário e a loja, pois eles aumentam o desempenho da loja, fazendo-a vender uma quantidade maior de produtos, e também facilitam a procura do usuário fazendo-o achar produto(s) desejados em um menor tempo\cite{isinkaye2015recommendation}. 

O primeiro sistema de recomendação foi criado na década de $90$ e tinha como nome ``filtragem colaborativa'', pois o sistema funcionava com base na colaboração entre os grupos de pessoas interessados. Contudo, o termo ``sistemas de recomendação'' é mais usado por ser mais geral, não sendo realizada, necessariamente, nenhuma colaboração entre pessoas\cite{reategui2005sistemas}. Já em 1996 o \textit{Yahoo} utilizou sistemas de recomendação em uma de suas páginas, aplicando em larga escala\cite{reategui2005sistemas}, coisa que hoje é feita comumente por diversos sites e serviços.

É facilmente perceptível no cotidiano o uso de sistemas de recomendação em ambientes on-line. Ao usar a \textit{Netflix}, sugestões para o usuário são oferecidas, baseadas nas atrações já assistidas e/ou avaliadas. Sites de compras como a \textit{Amazon} também oferecem sugestões de produtos ao usuário baseado em visitas à página dos produtos ou no comportamento de outros usuários que compraram um mesmo produto. Também em redes sociais, como no \textit{YouTube}, são sugeridos vídeos baseados no histórico do internauta e nas suas avaliações, ou então no \textit{Facebook}, que recomenda lista de pessoas que o usuário pode conhecer\cite{gorakala2015building}.

Em geral, sistemas de recomendação utilizam como informação a avaliação (\textit{rating}) dada pelos usuários aos itens, podendo a avaliação estar expressa de diferentes maneiras\cite{shapira2011recommender}:

\begin{itemize}

\item Avaliações numéricas: O usuário avalia um item numa escala numérica, como no site da \textit{Amazon}, onde o usuário dá uma nota de até 5 estrelas.

\item Avaliações qualitativas: A avaliação é dada por frases definidas, como: "Concordo totalmente", "Concordo parcialmente", ...

\item Avaliações binárias: O usuário seleciona se gostou ou não gostou do item, como a \textit{Netflix}, atualmente, recebe as avaliações.

\item Avaliação unária: A indicação se refere a se o usuário visualizou, comprou ou então avaliou o item positivamente.

\end{itemize}

\section{Técnicas de recomendação}

Existem diferentes categorias de sistemas de recomendação, que podem ser classificados em: Filtragem baseada em conteúdo (\textit{Content-based filtering}), filtragem colaborativa (\textit{Collaborative filtering}) e sistemas de recomendação híbridos (\textit{Hybrid Recommender Systems})\cite{melville2011recommender}.

\subsection{Filtragem baseada em conteúdo}

Os sistemas nesta categoria recomendam itens similares aos que o usuário gostou no passado\cite{gorakala2015building}. Para isto é necessário utilizar informações das características de um produto\cite{shapira2011recommender} e comparar com o perfil do usuário, de acordo com itens já conhecidos pelo usuário. Considerando filmes como itens, se um usuário avaliou positivamente filmes do gênero de ação, então o sistema recomendará a este usuário filmes de ação. Por outro lado, a filtragem baseada em conteúdo não leva em conta a similaridade de preferência entre os usuários, mas apenas o histórico do usuário e as características dos itens\cite{gorakala2015building}.

Algumas das técnicas utilizadas neste tipo de filtragem são: TF/IDF (\textit{Term Frequency Inverse Document}), \textit{naive Bayes Classifier}, árvores de decisão ou redes neurais\cite{isinkaye2015recommendation}. 

\subsection{Filtragem colaborativa}

Na filtragem colaborativa são recomendados itens de acordo com as avaliações de todos os usuários\cite{melville2011recommender}. Existem duas maneiras principais de realizar essa filtragem: baseado em memória ou em modelo\cite{dakhel2011new}. Nos algoritmos baseados em memória, verifica-se a similaridade entre usuários ou entre itens (vizinhança), de acordo com suas avaliações passadas. Essa técnica é a mais utilizada para realizar recomendações\cite{shapira2011recommender}. Um exemplo simples seria: Se o usuário 1 comprou o item A, B e C, e o usuário 2 comprou os itens A e C, então recomenda-se o item B para o usuário 2.

Os algoritmos de filtragem colaborativa utilizam uma matriz, chamada de matriz de avaliações (\textit{ratings matrix}), usualmente representada desta forma:

\begin{table}[!h]
\centering
\caption{Típica matriz $\mathbf{R}$ de avaliações}
\label{rating_matrix}
\begin{tabular}{c|c|c|c|c}
\hline \\
            & Item 1      & Item 2      & $\cdots$ & Item m      \\
Usuário 1   & $r_{(1, 1)}$ &             & $\cdots$  &             \\
Usuário 2   &  & $r_{(2, 2)}$ & $\cdots$ & $r_{(2, m)}$ \\
$\vdots$       &    $\vdots$  & $\vdots$       &  $\ddots$ & $\vdots$ \\
Usuário $n$ &  &             & $\cdots$ & $r_{(n, m)}$ \\
\hline
\end{tabular}
\end{table}

Onde $r_{(i, j)}$ é a avaliação (\textit{rating}) do usuário $i$ dado ao item $j$. Em geral, os usuários não tiveram contato com todos os itens, então os itens não recebem avaliações de todos os usuários, produzindo então uma matriz esparsa (com grande quantidade de valores faltantes). Os algoritmos buscam, então, preencher a matriz de avaliações com previsões para os valores faltantes.

À medida, porém, que os números de usuários e items aumentam, podem surgir problemas ao realizar a filtragem, como o aumento do tempo necessário, além de recursos computacionais, para executar o algoritmo, chamado de problema de escalabilidade\cite{dakhel2011new}. Além disso, existe o problema da esparsidade, pois um usuário, em geral, não avaliou uma grande quantidade de itens, mas apenas uma pequena quantidade, o que pode causar a impossibilidade do cálculo de medidas de similaridade (pois itens precisam ter sido avaliados por dois usuários), ou então pode levar, pela pequena quantidade de informação utilizada no cálculo da medida, a uma medida que não represente bem a real similaridade entre os usuários\cite{dakhel2011new}.

Buscando reduzir o tempo de processamento e melhores medidas de acurácia podem ser utilizados métodos de agrupamento (cluster)\cite{o1999clustering}. Uma possibilidade é agrupar usuários, de acordo com alguma informação disponível em $k$ clusters e, para cada um dos grupos de usuário, aplicar a técnica de recomendação.  

\subsection{Sistemas de recomendação híbridos}

Os sistemas híbridos são uma combinação da filtragem baseada em conteúdo e filtragem colaborativa, buscando aproveitar as vantagens e eliminar as desvantagens das técnicas\cite{shapira2011recommender}. Cada uma das técnicas podem ser aplicadas de maneira separada, combinando os resultados, mas também pode ser construído um modelo com as duas abordagens unificadas\cite{takahashi2015estudo}.

\chapter{Objetivos}

Este trabalho tem os seguintes objetivos:

\section{Objetivo geral}

Comparar a acurácia das recomendações utilizando filtragem colaborativa para todo o conjunto de dados com as recomendações utilizando filtragem colaborativa para cada cluster de usuários.

\section{Objetivos específicos}

\chapter{Materiais e Métodos}

\section{Conjunto de dados}

Será utilizado um \textit{dataset} disponível no site \textit{grouplens}, disponível em \burl{https://grouplens.org/datasets/movielens/1m/}. O conjunto de dados possui $\num{1000209}$ avaliações de $3900$ filmes dados por $6040$ usuários\cite{harper2016movielens}, que se cadastraram no site \textit{MovieLens} no ano de 2000. De acordo com o próprio site, pessoas podem se inscrever para avaliar filmes e receber recomendações de filmes para assistir.

Os usuários são representados pelo seu ID, que varia entre $1$ e $6040$ e os filmes possuem ID entre $1$ e $3952$. As avaliações têm formato numérico, de até 5 estrelas, com estrelas completas, tendo cada usuário avaliado ao menos 20 filmes. O conjunto de dados também apresenta o gênero dos filmes.

A base de dados será dividida em duas, treino e teste, na proporção de $70\%$ para treinar o modelo e $30\%$ que serão usados para que o modelo preveja as notas a fim de comparar com a nota real.

\section{Metodologia}

Haverão um conjunto de usuários $U = \{u_1, u_2, \ldots, u_n\}$ e um conjunto de itens $I = \{i_1, i_2, \ldots, i_m \}$, assim como as notas dos usuários aos itens, que serão armazenadas na matriz $\boldsymbol{R}_{n \times m}$ de avaliações\cite{hahsler2015recommenderlab}. Logo, cada linha da matriz $\boldsymbol{R}$ representa um usuário e cada coluna, um item. Os algoritmos buscarão preencher os valores faltantes desta matriz, com valores na mesma escala das avaliações presentes na matriz\cite{takahashi2015estudo}.

\subsection{Filtragem colaborativa baseada no item (item-based)}

Este algoritmo busca recomendar itens similares aos bem avaliados pelo usuário. Desta forma será verificado, para cada par de itens, a sua similaridade, e a partir desta medida é prevista a avaliação do usuário para o item. A similaridade entre dois itens $i$ e $j$ pode ser medida pelo coeficiente de correlação de Pearson, definido da seguinte maneira\cite{melville2011recommender}:

\begin{equation}
w_{i, j} = \frac{\sum_{u \in U}{(r_{u, i} - \overline{r}_i ) ( r_{u, j} - \overline{r}_j )}}{\sqrt{\sum_{u \in U}{(r_{u, i} - \overline{r}_i )^2} \sum_{u \in U}{(r_{u, j} - \overline{r}_j )^2}}}
\end{equation}

Sendo $U$ o conjunto de usuários que avaliaram os dois itens, $i$ e $j$,$r_{u, i}$ o rating dado pelo usuário $u$ ao item $i$ e $\overline{r}_i$ o rating médio recebido pelo item $i$ dado por todos os usuários que o avaliaram.

Alternativamente, a similaridade entre os itens $i$ e $j$ pode ser medida considerando os ratings recebidos pelos dois itens como vetores e calcular o cosseno entre estes vetores\cite{sarwar2001item}:

\begin{equation}
w_{i, j} = \cos({\vec{r}_i, \vec{r}_j}) = \frac{\vec{r}_i \boldsymbol{\cdot} \vec{r}_j}{\lVert \vec{r}_i\rVert \times \lVert \vec{r}_i\rVert} = \frac{\sum_{u=1}^{n}{r_{u, i} r_{u, j}}}{\sqrt{\sum_{u=1}^{n}{r^2_{u, i}} \sum{_{u=1}^{n}{r^2_{u, j}}}}}
\end{equation}

A seguir, o \textit{rating} do item $i$ pelo usuário $a$ pode ser previsto da seguinte forma\cite{melville2011recommender}:

\begin{equation}
p_{a, i} = \frac{\sum_{j \in k}{r_{a, i} - w_{i, j}}}{\sum_{j \in k}{\left|w_{i, j}\right|}}
\end{equation}

Sendo $k$ o conjunto de itens avaliados pelo usuário $a$ que são mais similares ao item $i$.

\subsection{Filtragem colaborativa baseada no usuário (user-based)}

Este algoritmo assume que usuários com preferência similar no passado terão preferências similares no futuro. Então os \textit{ratings} não observados serão previstos a partir das avaliações de uma vizinhança e usuários com gostos similares\cite{hahsler2015recommenderlab}. São então encontrados os $k$ vizinhos mais próximos de um usuário ou então todos os usuários que tenham pelo menos uma dada similaridade. O coeficiente de correlação de Pearson pode ser utilizado como medida de similaridade entre dois usuários $a$ e $u$, definida da seguinte maneira\cite{melville2011recommender}:

\begin{equation}
w_{a, u} = \frac{\sum_{i \in I}{(r_{a, i} - \overline{r}_a ) ( r_{u, i} - \overline{r}_u )}}{\sqrt{\sum_{i \in I}{(r_{a, i} - \overline{r}_a )^2} \sum_{i \in I}{(r_{u, i} - \overline{r}_u )^2}}}
\end{equation}

Sendo $I$ o conjunto de itens avaliados pelos dois usuários, $r_{u, i}$ é o rating dado pelo usuário $u$ ao item $i$ e $\overline{r}_u$ é o rating médio do usuário $u$ a todos os itens por ele avaliados. 

Uma outra maneira de calcular a similaridade entre dois usuários é considerar os ratings de dois usuários como vetores num espaço $m$-dimensional, para, assim, encontrar o cosseno do ângulo entre estes vetores\cite{melville2011recommender}:

\begin{equation}
w_{a, u} = \cos({\vec{r}_a, \vec{r}_u}) = \frac{\vec{r}_a \boldsymbol{\cdot} \vec{r}_u}{\lVert \vec{r}_a\rVert \times \lVert \vec{r}_u\rVert} = \frac{\sum_{i=1}^{m}{r_{a, i} r_{u, i}}}{\sqrt{\sum_{i=1}^{m}{r^2_{a, i}} \sum{_{i=1}^{m}{r^2_{u, i}}}}}
\end{equation}

Por fim, a predição da nota dada ao item $i$ pelo usuário $a$ é dada por:

\begin{equation}
p_{a, i} = \overline{r}_a + \frac{\sum_{u \in k}{(r_{u, i} - \overline{r}_u) w_{a, u}}}{\sum_{u \in k}{\left|w_{a, u}\right|}}
\end{equation}

Sendo $k$ a vizinhança do usuário $a$.

\subsection{\textit{PAM} (Partitioning Around Medoids)}

O algoritmo de agrupamento \textit{PAM} é baseado na definição de \textit{medoide}, que é o ponto com menor distância, em média, de todos os outros elementos do cluster. O algoritmo, para obter $k$ clusters, é executado da seguinte maneira\cite{do2005agrupamentos}:

\begin{enumerate}

\item{Definir aleatoriamente $k$ medoides.}
\item{Associar cada um dos elementos restantes a um cluster, sendo pertencente ao grupo de medoide mais próximo.}
\item{Calcular a dissimilaridade entre um elemento $x_i$ e todos os outros do cluster, e a dissimilaridade entre o medoide e os outros elementos do cluster.}
\item{Caso a distância considerando $x_i$ seja menor que a distância do medoide, passe a considerar $x_i$ como medoide daquele cluster.}
\item{Repetir os passos 2 a 4 até não haver troca de medoides.}

\end{enumerate}

Uma desvantagem desse método é a ineficiência ao ser aplicado para um grande conjunto de dados\cite{park2009simple}.

\subsection{\textit{CLARA} (Clustering Large Applications)}

Essa técnica foi proposta, em 1990, de forma a aplicar o PAM, solucionando o problema de escalabilidade, ao utilizar amostragem para a aplicação do PAM\cite{park2009simple}. O método, então, seleciona aleatoriamente uma parte da base de dados e aplica o algoritmo PAM nesta amostra. Em seguida é calculada a função de custo, que é uma média da similaridade entre os medoides e os outros elementos da base\cite{bhat2014k}. A função de custo é definida da seguinte maneira:

\begin{equation}
C(m, D) = \frac{\sum_{i=1}^{n}{d(x_i, cl(m, x_i))}}{n}
\end{equation}

Onde:

\begin{itemize}

\item{$m$ são os medoides encontrados.}
\item{$cl(m,  x_i)$ é o medoide mais próximo de um ponto $x_i$.}
\item{$d(x_i, cl(m, x_i))$ é uma medida de similaridade entre $x_i$ e seu medoide mais próximo.}
\item{$n$ é o número de observações na base de dados $D$.}

\end{itemize}

Todo o processo é repetido um número determinado de vezes e o resultado que obtiver menor função de custo é definido, então como o melhor e é retornado\cite{bhat2014k}.

\subsection{\textit{K-Means}}

K-means é uma técnica que particiona elementos em $k$ clusters utilziando-se de centroides, que são os elementos representativos de cada cluster. Este método busca minimizar a soma das distâncias dos elementos de um mesmo cluster. Dados então, uma matriz $D$, de dimensão $m \times n$, e um número de clusters $k$, o algoritmo, então, procede da seguinte maneira\cite{mining2006data}:

\begin{enumerate}

\item{São escolhidos, aleatoriamente, $k$ objetos de $D$ como sendo os centroides.}
\item{Cada elemento $D_{i,}$ é associado ao centroide mais próximo, de acordo com a medida de distância adotada (neste caso, a distância Euclidiana).}
\item{Os centroides de cada um dos clusters são calculados.}
\item{Repetir os passos 2 e 3 até que não haja mudanças.}

\end{enumerate}

\subsection{Medidas de erro}

Para verificar a acurácia do sistema de recomendação, dado que a base utilizada foi dividida em base de treino e de teste, serão comparadas a avaliação prevista e a avaliação observada. Considerando $r_{i,j}$ a avaliação observada e $p_{i,j}$ a avaliação prevista pelo modelo do usuário $i$ ao item $j$, as medidas utilizadas serão\cite{gorakala2015building}:

\subsubsection{Erro médio absoluto}

O erro médio absoluto (EMA) se dá pela soma do módulo das diferenças. 

\begin{equation}
EMA = \frac{1}{nm} \sum_{i=1}^{n} \sum_{j=1}^m \left| r_{i,j} - p_{i,j}  \right|
\end{equation}

\subsubsection{Erro quadrático médio}

O erro quadrático médio (EQM) é a soma das diferenças ao quadrado. Por este motivo, a unidade de medida muda, e a sua interpretação deve ser cautelosa.

\begin{equation}
EQM = \frac{1}{nm} \sum_{i=1}^{n} \sum_{j=1}^m \left( r_{i,j} - p_{i,j}  \right)^2
\end{equation}

\subsubsection{Raiz do erro quadrático médio}

Ao calcular a raiz do EQM obtém-se um número na mesma unidade de medida dos dados.

\begin{equation}
REQM = \sqrt{\frac{1}{nm} \sum_{i=1}^{n} \sum_{j=1}^m \left( r_{i,j} - p_{i,j}  \right)^2}
\end{equation}


% \section{Algumas Dicas}
% 
% Separe em seções para organizar melhor seu texto. Use os comandos \verb|\section| e \verb|\subsection| para isso. 
% 
% Sempre que preciso faça as referências usando o comando \verb|\cite|, como por exemplo, ``para mais informações sobre esse assunto veja o livro de Larson.''
% 
% Use os comandos a seguir para apresentar teoremas, definições, proposições, exemplos, etc. Dessa forma a numeração é feita de forma automática. 
% 
% {\defi
% Abra e feche as chaves e dentro delas coloque o comando \verb|\defi| seguido da definição. 
% }
% 
% {\thm
% Abra e feche as chaves e dentro delas coloque o comando \verb|\thm| seguido do enunciado do teorema. 
% }
% 
% {\prop
% Abra e feche as chaves e dentro delas coloque o comando \verb|\prop| seguido do enunciado da proposição.
% }
% 
% 
% {\exe
% Abra e feche as chaves e dentro delas coloque o comando \verb|\exe| seguido do exemplo.
% }

\chapter{Análise dos Resultados}



% Neste capítulo deve-se apresentar os resultados, podendo comentá-los e discuti-los, comparando-os com resultados da bibliografia já existente. Use seções, se apropriado.
% 
% Para que as tabelas e figuras sejam incluídas corretamente nas listas de tabelas e figuras use os comandos 
% \verb|\begin{table}...\end{table}| e \verb|\begin{figure}...\end{figure}|. Veja como nos exemplos a seguir.
% 
% Primeiro o exemplo de uma tabela. 
% 
% \begin{table}[h!]
% \centering
% \caption{Tabela exemplo} \label{fig:exemplo}
% \begin{tabular}{cc}
% \hline
% Ano & Produção (1.000t)\\
% \hline
% 1996 & 2.536\\
% 1997 & 2.666\\
% 1998 & 3.750\\
% 1999 & 2.007\\
% 2000 & 2.080\\
% \hline
% \end{tabular}
% \end{table}
% 
% Agora veja como incluir figuras.
% 
% \begin{figure}[h!]
% \centering
% \includegraphics[width=0.5\linewidth]{logoUFF.pdf}
% \caption{Figura exemplo} \label{tab:exemplo}
% \end{figure}
% 
% Use o comando \verb|ref| para fazer referências das tabelas e figuras ao longo do texto. Por exemplo, podemos nos referencias à Figura \ref{fig:exemplo} ou à Tabela \ref{tab:exemplo}. 
% 
% O mesmo comando \verb|ref| pode ser usado para fazer referência à capítulos, seções, equações, etc. Basta que tenha sido definido um \verb|label| para cada um deles. Por exemplo, se quisermos fazer referência ao número do capítulo Introdução basta digitar \verb|\ref{cap:introducao}|, veja como: ``... como comentado no Capítulo \ref{cap:introducao}, na introdução deste trabalho''. 

\chapter{Conclusão}

% Para finalizar seu trabalho faça o capítulo de conclusão. Neste capítulo deve ser feito um breve resumo de tudo que foi feito e as principais conclusões obtidas.
% 


%A bibliografia é feita automaticamente com o comando abaixo
%No arquivo referencias.bib devem ser incluídos todos os textos referenciados nesse trabalho.
\bibliographystyle{abnt-num} %abnt-num ou abnt-alf
\bibliography{referencias}


%Os anexos devem ser intorduzidos ao final do trabalho, depois das referências
\anexo



\chapter{Título do primeiro anexo}

% Caso você ache interessante, adicione anexos ao trabalho. Estes devem vir depois das referências bibliográficas. 



\chapter{Título do segundo anexo}

% Dessa forma você pode incluir quantos anexos quiser. 




\end{document}
