\documentclass[12pt,a4paper,header]{abnt}

\usepackage[brazil]{babel}        
\usepackage[utf8]{inputenc} 

\usepackage{hyperref}
\usepackage{bookmark}
\usepackage{amsmath,amssymb,amsfonts,undertilde}
\usepackage{graphicx}
\usepackage{subfigure}
\usepackage{fancyhdr}




%%%%%%%%%%%%%%%%%%%%%%%%%%%
%Teoremas, definicoes, etc%
%%%%%%%%%%%%%%%%%%%%%%%%%%%

\newtheorem{thm}{Teorema}[section]
\newtheorem{cor}[thm]{Corolário}
\newtheorem{lem}[thm]{Lema}
\newtheorem{defi}[thm]{Definição}
\newtheorem{exe}[thm]{Exemplo}
\newtheorem{prop}[thm]{Proposição}

\renewcommand{\ABNTchapterfont}{\bfseries}
\renewcommand{\ABNTsectionfont}{\bfseries}

\fancypagestyle{logouff}{%
	\renewcommand{\headrulewidth}{0pt}
	\fancyhead{}
	\fancyhead[R]{\includegraphics[width=0.7\textwidth]{logoUFF.pdf}}% Your logo/image
  	\setlength{\headheight}{30pt} 
  	\setlength{\headsep}{2cm}
}





\begin{document}

%%%%%%%%%%%%%%%%%%%%%%%%%%%%%%
%colocar aqui o nome do aluno%
%%%%%%%%%%%%%%%%%%%%%%%%%%%%%%
\autor{Leonardo Filgueira}


%%%%%%%%%%%%%%%%%%%%%%%%%%%%%%%%%%%%%
%colocar aqui o título da monografia%
%%%%%%%%%%%%%%%%%%%%%%%%%%%%%%%%%%%%%
\titulo{Sistemas de recomendação usando o software R}


%%%%%%%%%%%%%%%%%%%%%%%%%%%%%%%%%%%
%colocar aqui o nome do orientador%
%%%%%%%%%%%%%%%%%%%%%%%%%%%%%%%%%%%
\orientador{Luciane Ferreira Alcoforado}  


%%%%%%%%%%%%%%%%%%%%%%%%%%%%%%%%%%%%%%%%%%%%%%%%%%%
%colocar aqui o nome do co-orientador, caso exista%
%%%%%%%%%%%%%%%%%%%%%%%%%%%%%%%%%%%%%%%%%%%%%%%%%%%
\coorientador{Rodrigo Ribeiro}


%%%%%%%%%%%%%%%%%%%%%%%%%%%%%%%%%%%%%%%%%%%%%%%%%%%
%colocar aqui a data da apresentacao da monografia%
%%%%%%%%%%%%%%%%%%%%%%%%%%%%%%%%%%%%%%%%%%%%%%%%%%%
\data{30 de fevereiro de 2013}



%%%%%%%%%%%%%%%%%%%%%%%%%%%
%nao mexer até a linha 180%
%%%%%%%%%%%%%%%%%%%%%%%%%%%


\comentario{Monografia apresentada para obtenção do grau de Bacharel em Estatística pela Universidade Federal Fluminense.}

\instituicao{Departamento de Estatística \par Instituto de Matemática e Estatística \par Universidade Federal Fluminense}

\local{Niterói - RJ, Brasil}

\capa

\vspace{10cm}


%folha de rosto
%--------------

\begin{titlepage}

\thispagestyle{logouff}

\vspace{2cm}

\hspace{.2\textwidth} % posicionando a minipage
\begin{minipage}{.7\textwidth}

\begin{flushright}

{\large \bf \ABNTautordata} \\[3cm]

{\Large \bf \ABNTtitulodata}\\[3cm]

{\bf Trabalho de Conclusão de Curso}\\[1cm]

\end{flushright}

\begin{espacosimples}

\ABNTcomentariodata

\end{espacosimples}

\vspace{1cm}

\hfill Orientador: Prof. \ABNTorientadordata

\end{minipage}

\vspace{7cm}

\begin{center}

\ABNTlocaldata

\ABNTdatadata

\end{center}

\end{titlepage}


%ficha catalografica
%-------------------
\newpage
\null
\vfill

%\begin{left}
\fbox{
\includegraphics[scale=1]{ficha_catalografica.pdf}
}
%\end{center}
\vspace{1cm}

%folha de aprovacao com as assinaturas - Editar os membros da banca
%------------------------------------------------------------------
\begin{folhadeaprovacao}

\thispagestyle{logouff}

\hspace{.2\textwidth} % posicionando a minipage
\begin{minipage}{.7\textwidth}

\begin{flushright}

{\large \bf \ABNTautordata}\\[1cm]

{\large \bf \ABNTtitulodata}\\[1cm]

\end{flushright}

Monografia de Projeto Final de Graduação sob o título \textit{``\ABNTtitulodata''},
defendida por \ABNTautordata~e aprovada em \ABNTdatadata, na cidade de Niterói,
no Estado do Rio de Janeiro, pela banca examinadora constituída pelos
professores:

\begin{flushright}

\begin{espacosimples}

%assinatura




%%%%%%%%%%%%%%%%%%%%%%%%%%%%%%%%%%%%%%%%%
%Preencher os dados dos membros da banca%
%%%%%%%%%%%%%%%%%%%%%%%%%%%%%%%%%%%%%%%%%

\vspace{2cm}
\noindent\rule{8cm}{0.4pt}\\
{\bf Profa. Dra. Luciane Ferreira Alcoforado}\\
Departamento de Estatística -- UFF\\


\vspace{2cm}
\noindent\rule{8cm}{0.4pt}\\
{\bf Prof. Dr. Nome do 1o membro da banca}\\
Instituicao do 1o membro da banca\\


\vspace{2cm}
\noindent\rule{8cm}{0.4pt}\\
{\bf Profa. Me. Nome do 2o membro da banca}\\
Instituicao do 2o membro da banca\\

\end{espacosimples}

\end{flushright}

\vspace{2cm}
\hfill Niterói, \ABNTdatadata

\end{minipage}


%\vfill \hfill Niterói, \ABNTdatadata \hspace{1cm} 


\end{folhadeaprovacao}




%%%%%%%%%%%%%%%%%%%%%%%%%%%%%%%%%%%%%%%
%escreva aqui o resumo do seu trabalho%
%%%%%%%%%%%%%%%%%%%%%%%%%%%%%%%%%%%%%%%
\begin{resumo}
Aqui entra o resumo da monografia.

\vspace{1cm}
\noindent Palavras-chaves: 
%Aqui entram as palavras chaves.


\end{resumo}



%%%%%%%%%%%%%%%%%%%%%%%%%%%%%%%%
%escreva aqui a sua dedicatória% (opcional)
%%%%%%%%%%%%%%%%%%%%%%%%%%%%%%%%
\chapter*{Dedicatória}
Aqui entra a sua dedicatória. 



%%%%%%%%%%%%%%%%%%%%%%%%%%%%%%%%%%
%escreva aqui seus agradecimentos% (opcional)
%%%%%%%%%%%%%%%%%%%%%%%%%%%%%%%%%%
\chapter*{Agradecimentos}
Aqui entram os agradecimentos. 



\tableofcontents{}
\listoffigures
\listoftables



\chapter{Introdução} \label{cap:introducao}

A partir do aumento de informação disponível com a popularização da Internet e com a possibilidade de armazenar essas informações surge o desafio de lidar com este grande conjunto de dados\cite{isinkaye2015recommendation}. Este aumento de informações desafia o site, como a loja on-line, que recebe todas as informações dos usuários que visitam o endereço, mas também pode se tornar um problema para o usuário que, diante da grande quantidade de produtos disponíveis para compra, pode levar muito tempo para achar o produto desejado\cite{mild2002collaborative}.

Sistemas de recomendação são técnicas que filtram um grande conjunto de dados, tendo como base informações dos usuários\cite{takahashi2015estudo}. A partir dessas técnicas são previstas as notas que os usuários dariam a determinados itens, que podem ser dos mais variados tipos, e, para um indivíduo, recomenda-se o(s) item(ns) que obtiveram uma nota prevista maior\cite{shapira2011recommender}. Estes sistemas de recomendação beneficiam o usuário e a loja, pois eles aumentam o desempenho da loja, fazendo-a vender uma quantidade maior de produtos, e também facilitam a procura do usuário fazendo-o achar produto(s) desejados em menos tempo\cite{isinkaye2015recommendation}.


\section{Técnicas de recomendação}

Existem diferentes algoritmos de recomendação: Content-based filtering, Collaborative filtering e Hybrid Recommender Systems\cite{melville2011recommender}.
\chapter{Objetivos}

Este trabalho tem os seguintes objetivos:

\section{Objetivo geral}

\begin{itemize}

\item{Apresentar técnicas de sistemas de recomendação, executar algumas destas técnicas e avaliá-las.}

\end{itemize}

\section{Objetivos específicos}


\chapter{Materiais e Métodos}

Apresente aqui os materiais e métodos usados na monografia. 

Este capítulo pode ser modificado de acordo com o interesse do aluno e do orientador, dependendo do trabalho desenvolvido. Crie tantas seções quantas forem necessárias; a revisão bibliográfica pode também ser incluída aqui.


\section{Algumas Dicas}

Separe em seções para organizar melhor seu texto. Use os comandos \verb|\section| e \verb|\subsection| para isso. 

Sempre que preciso faça as referências usando o comando \verb|\cite|, como por exemplo, ``para mais informações sobre esse assunto veja o livro de Larson \cite{larson}.''

Use os comandos a seguir para apresentar teoremas, definições, proposições, exemplos, etc. Dessa forma a numeração é feita de forma automática. 

{\defi
Abra e feche as chaves e dentro delas coloque o comando \verb|\defi| seguido da definição. 
}

{\thm
Abra e feche as chaves e dentro delas coloque o comando \verb|\thm| seguido do enunciado do teorema. 
}

{\prop
Abra e feche as chaves e dentro delas coloque o comando \verb|\prop| seguido do enunciado da proposição.
}


{\exe
Abra e feche as chaves e dentro delas coloque o comando \verb|\exe| seguido do exemplo.
}




\chapter{Análise dos Resultados}


Neste capítulo deve-se apresentar os resultados, podendo comentá-los e discuti-los, comparando-os com resultados da bibliografia já existente. Use seções, se apropriado.

Para que as tabelas e figuras sejam incluídas corretamente nas listas de tabelas e figuras use os comandos 
\verb|\begin{table}...\end{table}| e \verb|\begin{figure}...\end{figure}|. Veja como nos exemplos a seguir.

Primeiro o exemplo de uma tabela. 

\begin{table}[h!]
\centering
\caption{Tabela exemplo} \label{fig:exemplo}
\begin{tabular}{cc}
\hline
Ano & Produção (1.000t)\\
\hline
1996 & 2.536\\
1997 & 2.666\\
1998 & 3.750\\
1999 & 2.007\\
2000 & 2.080\\
\hline
\end{tabular}
\end{table}

Agora veja como incluir figuras.

\begin{figure}[h!]
\centering
\includegraphics[width=0.5\linewidth]{logoUFF.pdf}
\caption{Figura exemplo} \label{tab:exemplo}
\end{figure}

Use o comando \verb|ref| para fazer referências das tabelas e figuras ao longo do texto. Por exemplo, podemos nos referencias à Figura \ref{fig:exemplo} ou à Tabela \ref{tab:exemplo}. 

O mesmo comando \verb|ref| pode ser usado para fazer referência à capítulos, seções, equações, etc. Basta que tenha sido definido um \verb|label| para cada um deles. Por exemplo, se quisermos fazer referência ao número do capítulo Introdução basta digitar \verb|\ref{cap:introducao}|, veja como: ``... como comentado no Capítulo \ref{cap:introducao}, na introdução deste trabalho''. 




\chapter{Conclusão}

Para finalizar seu trabalho faça o capítulo de conclusão. Neste capítulo deve ser feito um breve resumo de tudo que foi feito e as principais conclusões obtidas.



%A bibliografia é feita automaticamente com o comando abaixo
%No arquivo referencias.bib devem ser incluídos todos os textos referenciados nesse trabalho.
\bibliographystyle{abnt-num} %abnt-num ou abnt-alf
\bibliography{referencias}


%Os anexos devem ser intorduzidos ao final do trabalho, depois das referências
\anexo



\chapter{Título do primeiro anexo}

Caso você ache interessante, adicione anexos ao trabalho. Estes devem vir depois das referências bibliográficas. 



\chapter{Título do segundo anexo}

Dessa forma você pode incluir quantos anexos quiser. 




\end{document}
